\chapter{格式化文本和创建宏}
\section{使用逻辑格式}
在 \LaTeX 文档中,我们不应该应用物理格式(physical formatting),例如将单词设为粗体或斜体或使用不同的大小。相反,我们应该使用逻辑格式(logical formatting),例如声明标题和作者并给出章节标题。实际的格式,例如用大写字母打印标题和将章节标题设为粗体,都是由 \LaTeX 完成的。

在一份好的 \LaTeX 文档中,物理格式仅在逻辑格式命令的定义范围内使用。如果我们需要某种格式样式,例如关键字,我们将在文档序言中定义合适的逻辑命令。在文档正文中,我们只应使用逻辑格式命令。这可使我们在整个文本中保持一致的格式,并且每当我们改变对格式细节的想法时,我们都可以修改序言中的逻辑命令。
\subsection{探索文档结构}
通常,\LaTeX 文档并不是独立的,文档基于一个多功能模板。这种基本模板称为类(class)。它提供可自定义的功能,通常为特定目的而构建。有针对书籍、期刊文章、信件、演示文稿、海报等的类;互联网档案中可以找到数百个可靠的类,安装 TeXLive 后,也可以在计算机上找到。

以反斜杠开头的单此被称为命令(command)或宏(macro)。文档的第一部分称为文档的序言。我们在这里选择类、指定属性,并通常进行文档范围的定义。

\verb|\begin{document}| 标记序言的结束和实际文档的开始。 \verb|\end{document}| 标记文档的结束。后面的所有内容都会被 \LaTeX 忽略。 通常,由 \verb|\begin| 和 \verb|\end|命令框起来的一段代码被称为一个环境(environment)。

在实际文档中,我们使用 \verb|\maketitle| 命令打印标题、作者和日期。
\subsection{理解 \LaTeX 命令}
\LaTeX 命令以反斜杠开头,后跟大写或小写字母,通常以描述性方式命名。但也有例外:一些命令仅由一个反斜杠和一个特殊字符组成。

命令可以有参数,即决定命令以何种方式工作的选项。我们作为参数传递的值称为参数。它们用花括号或方括号给出。

调用命令可以像这样:

\begin{verbatim}
    \command[optional argument]{argument}
\end{verbatim}
\subsection{了解 \LaTeX 环境}
\LaTeX 环境以 \textbackslash begin 开始,以 \textbackslash end 结束。这两个命令都需要环境名称作为其参数。与命令一样,环境也可以有参数。与命令的情况完全一样,强制参数写在花括号中,可选参数写在方括号中。
\section{修改文本字体}
文本格式化命令通常类似于 \textbackslash text**\{argument\},其中 ** 代表两个字母的缩写,例如 bf 代表粗体,it 代表斜体,sl 代表倾斜。
\subsection{通过括号限制命令的效果}
\textbackslash normalfont 来将字体切换回默认字体,但还有另一种方法。我们将使用花括号来告诉 \LaTeX 在哪里应用命令以及在哪里停止它。

当我们使用声明更改字体时,我们从左花括号开始,然后是字体声明命令。该命令的效果会持续到我们用相应的右花括号结束它为止。左花括号指示 \LaTeX 开始一个组。以下命令对后续文本有效,直到右花括号结束该组,组可以嵌套。简而言之,组由花括号定义,它们包含和限制局部命令的效果。

\subsection{探索字体大小}